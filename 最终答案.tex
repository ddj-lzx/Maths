\documentclass{article}
\usepackage[UTF8]{ctex}
\usepackage{amsmath, amssymb}
\title{正多边形内生成给定角度最大数目的通项公式}
\author{DDJ_LzX}
\date{}
\begin{document}
\maketitle
设正$n$边形的内角为$\alpha = 180^\circ - \frac{360^\circ}{n}$.定义辅助量:
\begin{itemize}
\item 内部交点贡献$C_{\text{int}}(\theta, l)$,
\item 边界交点贡献$C_{\text{edge}}(\theta, n, l)$,
\item 多边形顶点贡献$C_{\text{vertex}}(\theta, n)$.
\end{itemize}
则最多能得到度数为$\theta$的角的个数为
\[
M(n, l, \theta) = C_{\text{int}} + C_{\text{edge}} + C_{\text{vertex}}.
\]
其中:
\[
C_{\text{int}}(\theta, l) = 
\begin{cases}
4\left\lfloor\dfrac{l^2}{4}\right\rfloor, & \theta = 90^\circ,\\[1.2em]
2\left\lfloor\dfrac{l^2}{3}\right\rfloor, & \theta = 60^\circ \text{ 或 } 120^\circ,\\[1.2em]
2\left\lfloor\dfrac{l^2}{4}\right\rfloor, & \text{其他}.
\end{cases}
\]
\[
C_{\text{edge}}(\theta, n, l) = 
\begin{cases}
4l, & \theta = 90^\circ \text{ 且 } n \text{ 为偶数},\\
2l, & \theta = 90^\circ \text{ 且 } n \text{ 为奇数},\\[1em]
2l, & \theta \neq 90^\circ \text{ 且 } (2\theta \bmod 180) \text{ 是 } \delta \text{ 的整数倍},\\
l, & \theta \neq 90^\circ \text{ 且 } (2\theta \bmod 180) \text{ 不是 } \delta \text{ 的整数倍},
\end{cases}
\]
其中
\[
\delta = 
\begin{cases}
\dfrac{180^\circ}{n}, & n \text{ 为奇数},\\[1em]
\dfrac{360^\circ}{n}, & n \text{ 为偶数}.
\end{cases}
\]
\[
C_{\text{vertex}}(\theta, n) = n \cdot \mathbf{1}_{\theta = \alpha},
\]
这里$\mathbf{1}$为指示函数,且仅当$\alpha$为整数时可能取非零值.
此公式综合了内部平行线组的最优配置,边界交点的最大化条件以及多边形原有顶点的贡献.注意,$\theta$为整数角度,且$0^\circ<\theta<180^\circ$.
\end{document}